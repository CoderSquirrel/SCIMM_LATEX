%====================================================================================================
% ?????
%====================================================================================================
% TCC
%----------------------------------------------------------------------------------------------------
% Autor				: Jasane Schio
% Orientador		: Gedson Faria
% Co-Orientador		: Angelo Darcy
% Instituição 		: UFMS - Universidade Federal do Mato Grosso do Sul
% Departamento		: CPCX - Sistema de Informação
%----------------------------------------------------------------------------------------------------
% Data de criação	: 01 de Outubro de 2015
%====================================================================================================

\chapter*{Abstract}

Due to some aspects as light sources, shadows and luminosity, a single color can present itself as many diferentes ways inside the same image. In robot soccer, category Very Small Size Soccer, this problem also occurs once the robots are detected through a camera and recognized by the color mark above them. Therefore a challenge in robot soccer is to implement a digital image processing to object detection. This work proposes an automatic color calibration system, reducing time spent by the robot soccer's teams in this task. Was used the Canny algorithm technique, which makes the detection of the edges present in the image and right after uses up the border following algorithm, both present in the OpenCV library. As a result the colors green and orange had 100 \% success rate in detection, yellow and blue colors were successfully detected with error percentage of 6.66 \%, the color purple had found all the objects and 26.66 \% of them completely, the color red was successful detection of objects with their error rate of 19.99 \%, and the color Pink was not successfully detected.

Keywords: Color Calibration, Robot Soccer, OpenCV, Computer Vision, HSV, Team Cedro;