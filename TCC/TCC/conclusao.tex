%====================================================================================================
% ?????
%====================================================================================================
% TCC
%----------------------------------------------------------------------------------------------------
% Autor				: Jasane Schio
% Orientador		: Gedson Faria
% Co-Orientador		: Angelo Darcy
% Instituição 		: UFMS - Universidade Federal do Mato Grosso do Sul
% Departamento		: CPCX - Sistema de Informação
%----------------------------------------------------------------------------------------------------
% Data de criação	: 01 de Outubro de 2015
%====================================================================================================

\chapter{Conclusão} \label{Cap:Conclusao}


	Foi realizado um estudo dos modelos de cores dentro da biblioteca OpenCV e foram identificados intervalos definidos de cada uma das cores. Seus espectros foram encontrados uma vez que se identificou que, ao contrario do que acontece no mundo real, um pixel na biblioteca assume o  padrão GBR e uma vez convertido ao HSV possui os valores alterados. Assim por meio de experimentação e da ajuda de artigos já disponíveis na internet, se chegou ao melhor espectro para cada cor, categorizando o pixel dentre seus respectivos intervalos. Para serem identificadas as faixas a serem calibrados no campo, utilizou-se a técnica Algoritmo de Canny, que faz a detecção das bordas presentes na imagem. Após serem detectadas as bordas fez-se então uma analise para identificar quais bordas pertencem à mesma faixas, é somente após essa analise que as faixas são identificas pela junção de suas respectivas bordas. Assim que identificadas, cada um dos faixas foi analisado de acordo com o estudo das cores para assimilar a qual cor a mesma pertencia, e seus valores comparados para identificar se esta seria um limite do intervalo da cor a qual pertence.

O teste foi separado em \textbf{Calibração} e \textbf{Teste}. Para ambos foram usadas 7 cores: Amarelo, Azul, Laranja, Rosa, Roxo, Verde e Vermelho. A Calibração durou ao todo aproximadamente 5 minutos e gerou o arquivo \textit{cores.arff} contendo os valores HSV mínimo e máximo para cada cor. Para o Teste foi desenvolvida uma aplicação que faz a exibição das cores de acordo com seus valores no arquivo \textit{cores.arff}. A aplicação mostra uma imagem somente com os objetos da cor selecionada, esta imagem é a que foi usada nos testes do Capitulo 4. Cada objeto de cada cor foi categorizado em: Completo; Com Falha de Preenchimento; Com Diminuição de Contorno; Com Diminuição de Área ou Com Falha Criticas. Das categorias de objetos citadas, as que possuem a área do objeto deformada, o que dificulta sua identificação, são \textit{Falha de Preenchimento} e \textit{Falha Criticas}. Somente tr\^{e}s cores apresentaram estas falhas: Amarelo, com 6,66\%; Vermelho, com 13,33\%; e Rosa, com 26,66\%. De outro modo, somente uma cor não apresentou objetos completos, a cor Rosa. As cores Amarelo, Azul, Verde e Laranja obtiveram mais de 90\% de seus objetos encontrados completamente. A cor Vermelha obteve 53,33\% de seus objetos encontrados e o Roxo 26,66\%. 
Portanto o sistema cobre seu objetivo principal de automatizar o processo de calibração de cores, além da diminuição do tempo de calibração, uma vez que o sistema faz a calibração de todas as cores ao mesmo tempo, o que era feito cor por cor da forma manual.	

\section{Trabalhos Futuros}
Melhorias são indicadas para o sistema tanto na área de detecção de objetos quanto para a categorização de cores. Na detecção de objetos recomenda-se aprimorar a técnica para que não necessitem da subtração de fundo, pois a técnica utilizada neste trabalho infelizmente requer um certo tempo de processamento bem como ação manual, assim sera possível que o sistema se torne 100\% automatizado e ainda mais rápido.

 Quanto a categorização de cores: a aplicação da técnica também para a cor ciano, bem como melhorar os valos de S e V, uma vez que esses estão sendo usados de maneira estática, seria indicado uma melhoria para que os mesmos sejam os valores detectados das tiras de cor, o que necessita de um certo tratamento.
 
Outra indicação de incrementação do projeto seria a conversão do sistema para uma biblioteca modular, para que o mesmo possa ser incorporado em diferentes sistemas de estrategia. 