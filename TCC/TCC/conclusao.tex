%====================================================================================================
% ?????
%====================================================================================================
% TCC
%----------------------------------------------------------------------------------------------------
% Autor				: Jasane Schio
% Orientador		: Gedson Faria
% Co-Orientador		: Angelo Darcy
% Instituição 		: UFMS - Universidade Federal do Mato Grosso do Sul
% Departamento		: CPCX - Sistema de Informação
%----------------------------------------------------------------------------------------------------
% Data de criação	: 01 de Outubro de 2015
%====================================================================================================

\chapter{Conclusão} \label{Cap:Conclusao}


Foi estudado o problema da calibração de cores para competição de futebol de robôs categoria Very Small Size Soccer. Onde a calibração de cada uma das cores em campo acaba se tornando um processo euxastivo por ter de ser feito um a um, cerca de 5 minutos para cada cor. Geralmente o tempo de preparo inicial antes de cada jogo é de 20 minutos, tempo esse que acaba sendo usado praticamente inteiro no processo de calibração. Se o mesmo fosse automatizado, e assim reduzido, haveria mais tempo para ser usado em melhorias tecnicas, hardware dos robos com problemas, melhorias na estratégia de jogo, ou resolvendo problema de comunicação.

	Foi realizado um estudo dos modelos de cores dentro da biblioteca OpenCV e foram identificados intervalos definidos de cada uma das cores. Seus espectros foram encontrados uma vez que se identificou que, ao contrario do que acontece no mundo real, um pixel na biblioteca assume o  padrão GBR e uma vez convertido ao HSV possui os valores alterados. Assim por meio de experimentação e da ajuda de artigos já disponiveis na internet, se chegou ao melhor espectro para cada cor, categorizando o pixel dentre de seus respectivos intervalos.Para a serem identificados os objetos cores a serem calibrados no campo, utlizou-se da tecnica de conhecida como Algoritmo de Canny, que faz a detecção das bordas presentes na imagem. Após serem detectadas as bordas fez-se então uma analise para identificar quais bordas pertencem ao mesmo contorno objeto, contorno, e é somente após essa analise que os objetos são formados pela junção de suas respectivas bordas. Assim que identificado, cada um dos objetos foi analisado de acordo com o estudo das cores para assimilar a qual cor o mesmo pertencia e seus valores comparados para identificar se este seria um limite do intervalo da cor a qual pertence.

Para realização do teste foram usadas 7 cores no processo de calibração: Amarelo, Azul, Laranja, Rosa, Roxo, Verde e Vermelho. O teste foi separado em \textbf{Calibração} e \textbf{Teste}. A Calibração durou ao todo aproximadamente 5 minutos e gerou o arquivo \textit{cores.arff} contendo os valores HSV minimo e maximo para cada cor. 
Para os testes foi desenvolvida uma aplicação que faz a exibição das cores de acordo com seus valores no arquivo \textit{cores.arff}. A aplicação mostra uma imagem somente com os objetos da cor selecionada, esta imagem é a que foi usada nos testes do Capitulo 4. Cada objeto de cada cor foi categoriazado em: Completo; Com Falha de Preenchimento; Com Diminuição de Contorno; Com Diminuição de Área ou Com Falha Criticas. 
Das categorias de objetos citadas, as que descaracterizam o mesmo para detecção baseando na sua área são \textif{Com Falha de Preenchimento} e \textit{Com Falha Criticas}.


Das 7 cores, 4 obtiveram sua taxa de Objetos Completos maior que 90\%, foram elas Amarelo, Azul, Verde e Laranja; 1 obteve a mesma taxa acima de 50\%, a cor Vermelha; A cor Roxo obteve sua taxa de objetos completos acima de 20\% e a cor Rosa foi a unica em que não foi possivel identificar nenhum Objeto Completo.
