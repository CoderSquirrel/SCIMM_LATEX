%====================================================================================================
% ?????
%====================================================================================================
% TCC
%----------------------------------------------------------------------------------------------------
% Autor				: Jasane Schio
% Orientador		: Gedson Faria
% Co-Orientador		: Angelo Darcy
% Instituição 		: UFMS - Universidade Federal do Mato Grosso do Sul
% Departamento		: CPCX - Sistema de Informação
%----------------------------------------------------------------------------------------------------
% Data de criação	: 01 de Outubro de 2015
%====================================================================================================

\chapter{Conclusão} \label{Cap:Conclusao}

\section{Principais Considerações}
	Foi realizado um estudo dos modelos de cores dentro da biblioteca OpenCV e identificado que o mesmo não possuía intervalos definidos no padrão HSV. Percebeu-se que uma simples conversão de valores seguindo a teoria das cores não atenderia a necessidade do sistema, sendo que a biblioteca utiliza pixel no formato GBR e não RGB. Deste modo por meio de experimentação e com a ajuda de fóruns especializados, foi definido espectro para cada cor, categorizando o pixel dentre seus respectivos intervalos.
	
	Outro aspecto identificado durante o desenvolvimento deste trabalho foi quanto os objetos coloridos usados durante a calibração. Como percebido durante a participação da equipe CEDRO no LARC 2015, a maioria das equipes utilizaram os próprios robôs ou quadrados de aproximadamente \textit{4cm x 4cm} o que foi identificado neste trabalho que dificultava a detecção e por vezes diminuía o espectro de cores. Por este motivo foi usado tiras de cores com tamanho entre 4 e 10 vezes maior que o utilizado anteriormente. A disposição das tiras no campo também é fundamental no processo de calibração, foi usado uma disposição de 3 tiras por cor, estado elas em cada canto, bem como no meio. 
	
	O sistema atingiu satisfatoriamente o seu objetivo principal de diminuir o tempo do processo de calibração, tendo reduzido a duração do processo de calibração para apenas um quarto do tempo de \emph{aquecimento}, bem como a automatização da calibração, pois o sistema encontra sozinho os objetos em campo e os categoriza na sua cor adequada. 

A avaliação dos resultados foi separado em \textbf{Calibração} e \textbf{Teste}. Para ambos foram usadas 7 cores: Amarelo, Azul, Laranja, Rosa, Roxo, Verde e Vermelho. A Calibração durou ao todo aproximadamente 5 minutos e gerou o arquivo \textit{cores.arff} contendo os valores HSV mínimo e máximo para cada cor. Para o Teste foi desenvolvida uma aplicação que faz a exibição das cores de acordo com seus valores no arquivo \textit{cores.arff}. A aplicação mostra uma imagem somente com os objetos da cor selecionada, esta imagem é a que foi usada nos testes do Capitulo 4. Cada objeto de cada cor foi categorizado em: Completo; Com Falha de Preenchimento; Com Diminuição de Contorno; Com Diminuição de Área ou Com Falha Criticas. Das categorias de objetos citadas, as que possuem a área do objeto deformada, o que dificulta sua identificação, são \textit{Falha de Preenchimento} e \textit{Falha Criticas}. Somente tr\^{e}s cores apresentaram estas falhas: Amarelo, com 6,66\%; Vermelho, com 13,33\%; e Rosa, com 26,66\%. De outro modo, somente uma cor não apresentou objetos completos, a cor Rosa. As cores Amarelo, Azul, Verde e Laranja obtiveram mais de 90\% de seus objetos encontrados completamente. A cor Vermelha obteve 53,33\% de seus objetos encontrados e o Roxo 26,66\%. 
Analisando os resultados obtidos pelo sistema, aconselha-se que seja usadas as cores Vermelho, Verde e Roxo para designar os membros de equipe. Quando a cor da equipe, mesmo que tanto Amarelo quando Azul tenham obtido valores considerados bons, a cor Azul ainda se sobressaiu à Amarela, sendo então indicada para escolha, ao usar o sistema. 

\section{Trabalhos Futuros}
Melhorias são indicadas para o sistema tanto na área de detecção de objetos quanto para a categorização de cores. Na detecção de objetos recomenda-se aprimorar a técnica para que não necessite da subtração de fundo, pois a técnica utilizada neste trabalho infelizmente requer um certo tempo de processamento bem como ação manual, assim sera possível que o sistema se torne 100\% automatizado e ainda mais rápido.

 Quanto a categorização de cores: a aplicação da técnica também para a cor ciano, bem como melhorar os valos de S e V, uma vez que esses estão sendo usados de maneira estática, seria indicado uma melhoria para que os mesmos sejam os valores detectados das tiras de cor, o que necessita de um certo tratamento.
 
Outra indicação de incrementação do projeto seria a conversão do sistema para uma biblioteca modular, para que o mesmo possa ser incorporado em diferentes sistemas de estrategia. 