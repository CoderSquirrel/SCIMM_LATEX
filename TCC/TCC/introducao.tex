%====================================================================================================
% ?????
%====================================================================================================
% TCC
%----------------------------------------------------------------------------------------------------
% Autor				: Jasane Schio
% Orientador		: Gedson Faria
% Co-Orientador		: Angelo Darcy
% Instituição 		: UFMS - Universidade Federal do Mato Grosso do Sul
% Departamento		: CPCX - Sistema de Informação
%----------------------------------------------------------------------------------------------------
% Data de criação	: 01 de Outubro de 2015
%====================================================================================================
% NO FUTURO
\chapter{Introdução} \label{Cap:Introducao}
\section{Considerações Iniciais}
Em 1997 foi estabelecida a Federation of International Robot-soccer Association (FIRA), com o intuito de promover o desenvolvimento nas áreas de multi-agentes autonomos e cooperação entre robôs, bem como pesquisas e estudos relacionados a mecatrônica, processamento de imagens e robótica\cite{FiraHistory,FiraOverview}. O Futebol de Robôs tem sido  uma das principais áreas de foco por ser um domínio complexo, exigindo autonomia do sistema e solução de problemas em tempo real\cite{Costa:2000,Faria2006}. 

Nos jogos de futebol, o técnico passa as informações pessoalmente para seus jogadores em campo, porem isso não é possível no futebol de robôs, já que neste ambiente tais informações precisam ser passadas via comunicação de dados.%, no qual cada um dos robôs se difere do outro por obter um identificador único. 
Os robôs são diferenciados computacionalmente por um identificador único (e.g. endereço MAC) e em sua carcaça por marcadores com duas cores, uma cor designando o time a qual o robô pertence e outra que o identifica de maneira única dentro de sua equipe. O software de estratégia de cada equipe deve detectar os marcadores de cor utilizando técnicas de visão computacional e decidir a partir da posição dos robôs em campo qual atitude será realizada.

Para identificar cada uma das cores devem ser considerados alguns fatores como: a iluminação local, a tonalidade de cor dos marcadores, a posição dos refletores e sombras sobre o campo, portanto, é necessário que se faça o processo chamado de calibração. A calibração de cores ocorre para designar o intervalo de valores que corresponde a cada cor naquele determinado momento. Intervalos de cores podem ser pré definidos, no entanto, de acordo a mudança de qualquer um dos fatores já citados, uma nova calibração se faz necessária.

Neste trabalho, foi estudado o problema da calibração de cores para competição de futebol de robôs categoria {\it IEEE Very Small Size Soccer} (VSSS), na qual a calibração de cada uma das cores em campo acaba se tornando um processo exaustivo por ter de ser feito um a um, cerca de 5 minutos para cada cor. Sendo assim, automatizar o processo de calibração, auxiliaria em um problema comum a todas as equipes. Na próxima seção, serão apresentados dois trabalhos exemplificando a calibração de forma manual.  

\section{Trabalhos Correlatos}
Na pesquisa sobre calibração de intervalo de cores para times de futebol de robôs, não foram encontrados trabalhos que descrevessem com detalhes a implementação do sistema de calibração, sendo os mais completos o {\it Team Discription Papers}
\footnote{Documento obrigatório para inscrição nas competições brasileiras que descreve o time de robôs, contendo informações sobre o chassi, eletrônica, visão computacional e estratégia.} 
dos times.

\subsection{Calibra}
O Centro Universitário da Fundação Educacional Inaciana(FEI) propôs o sistema denominado CALIBRA, desenvolvido para sistemas Linux e com {\it Graphical User Interface}. O sistema de calibração possui um módulo chamado de {\it MainWindow} que disponibiliza a configuração de brilho, cor e contraste da imagem adquirida pela câmera, e grava estas configurações em um arquivo que é utlizado no momento de definir os intervalos no espaço de cores HSI para cada uma das cores dos marcadores de identificação dos robôs\cite{Penharbel:2004,PenharbelTime}. Deve-se ressaltar que a descrição do sistema CABLIBRA não contempla os detalhes de implementação da calibração das cores.

\subsection{VSS-Vision}

O VSS-Vision é um sistema completo de controle da equipe de futebol de robôs da categoria VSSS do Laboratório de Sistemas Inteligentes e Robótica (SIRLab) da Faeterj/Petrópolis  \cite{Rosa:2015}. %Este sistema de contempla a calibração de cores e foi utilizado durante a competição XXXXY (LARC) do ano de 2014.

\citeonline{Rosa:2015} menciona que a calibração de cores é feita calibrando obrigatoriamente laranja, amarelo e azul, e então as outras cores referentes aos jogadores em campo. A imagem da câmera é dividida em nove partes, e para calibrar a cor o usuário deve clicar sobre a cor que gostaria de ser calibrada, assim salvando um intervalo de cor tratado como RGB máximo e o mínimo daquela cor, a medida
que vão havendo os cliques, o sistema verifica para cada atributo (R,G e B) se ele é maior que o atributo
máximo salvo ou menor que mínimo salvo, caso seja, o mesmo assume o lugar de menor ou
maior, e esse processo deve ser feito em cada uma das nove partes da imagem. 
%Os valores HSV encontrados s\~ao ajustados manualmente com a ajuda de sliders, como visto na Figura \ref{SIRLabCalibracaoHSV}. Este processo de calibração pode demorar entre cinco e dez minutos.
Para desenvolvimento do sistema de processamento de imagens foi utilizada a biblioteca OpenCV e para telas interativas a biblioteca ImGui.

O atual sistema de visão computacional do SIRLab passou por algumas mudanças desde 2015 e conta com uma inteface e método de calibração diferentes do inicial, utilizando-se da calibração no espaço de cores HSV, no lugar do RGB. A antiga interface do sistema, feita inicialmente em ImGui deu lugar a nova, desenvolvida em Qt\cite{VSSVision}.

O método de calibração de cores também foi modificado possibilitando a calibração de 8 cores: laranja, amarelo, azul, vermelho, verde, rosa, roxo e marrom\cite{VSSVision}. O processo de calibração das cores ocorre de forma manual, primeiramente o usuário escolhe uma cor e clica em algum objeto desta cor, o sistema então reconhece os valores  mínimos e máximos de HSV. No segundo passo, o usuário realiza um ajuste fino em cada um dos atributos (H, S e V) até encontrar os valores adequados que representem não só objeto selecionado, mas todos os objetos da cor escolhida.

\section{Motivação e Justificativa}
Em 2015 surgiu a equipe Cedro, equipe de futebol de robôs categoria VSSS da UFMS Câmpus Coxim. Neste mesmo ano, a equipe participou da Competição Latino Americana de Robótica (LARC), na qual foram realizadas algumas observações sobre as demais equipes e as partidas, mais especificamente, sobre a calibração de cores, motivando os estudos e pesquisas presentes neste trabalho. Embora não conste nas regras oficiais, todo jogo possue um tempo de 20 minuntos para aquecimento, no qual a maioria das equipes realiza a calibração de cores. Nas equipes observadas, o processo de calibração foi feito de forma manual com uma interface muito semelhante ao deselvolvido por Kyle Hounslow\cite{YouTube}, sendo que, a equipe POTI/UFRN desenvolveu em 2007 um novo modelo de cores, o HPG, o qual utiliram em todas as competições desde então  \cite{Martins:2007,Mendes:2008}.

%Geralmente o tempo de preparo inicial antes de cada jogo é de 20 minutos, tempo que acaba sendo gasto praticamente inteiro no processo de calibração. 
Automatizar o processo de calibração das cores, reduziria o tempo utilizado, e assim, haveria mais tempo para ser usado em melhorias técnicas, hardware dos robôs com problemas, melhorias na estratégia de jogo ou resolvendo problema de comunicação. Assim, o beneficio trazido para as equipes justifica o desenvolvimento um sistema autônomo de registro do valores HSV, que faça a definição de intervalos de cores baseando-se nos objetos em campo, os identificando de forma automática, e assim reduzindo o tempo gasto na calibração de cores. 
\section{Objetivos}
Neste trabalho tem-se por objetivo principal automatizar o sistema de calibração de cores no espaço HSV em imagens capturadas em tempo real, identificando os limites mínimos e máximos dos atributos H, S e V para as cores de identificação dos robôs.  
Para desenvolver um sistema que supra essas necessidades, foram propostos os seguintes objetivos específicos:

\begin{itemize}
	\item Estudar detecção de bordas e implementar a detecção automática dos objetos em campo; 
	\item Estudo do espaço de cores RGB, HSV e as diferenças para BGR e HSV presentes na biblioteca {\it Open Source Computer Vision Library}(OpenCV);
	\item Analisar as cores dos objetos detectados e classifica-los no espaço de cores HSV do OpenCV;
	\item Testar a eficiência do sistema de calibração automático no reconhecimento de objetos estáticos em diversas localizações do campo.
	
	
\end{itemize}

%\newpage
\section{Organização da Trabalho} \label{Sec:Organizacao}

Este trabalho está divido em cinco capítulos. No segundo capítulo encontra-se a fundamentação teórica, contento informações sobre processamento de imagens referente à detecção de objetos e cores e uma breve descrição sobre o futebol de robôs. No terceiro capítulo encontra-se todo o desenvolvimento do projeto, suas classes, a descrição das tecnologias utilizadas no projeto, e a descrição do processo de calibração. No quarto capítulo são apresentados os testes e discutidos os resultados obtidos. No quinto capítulo são feitas as conclusões do trabalho, bem como expostas melhorias que possam vir a ser implementadas em trabalhos futuros.

%\section{Considerações Finais}
