%====================================================================================================
% ?????
%====================================================================================================
% TCC
%----------------------------------------------------------------------------------------------------
% Autor				: Jasane Schio
% Orientador		: Gedson Faria
% Co-Orientador		: Angelo Darcy
% Instituição 		: UFMS - Universidade Federal do Mato Grosso do Sul
% Departamento		: CPCX - Sistema de Informação
%----------------------------------------------------------------------------------------------------
% Data de criação	: 01 de Outubro de 2015
%====================================================================================================

\definecolor{dkgreen}{rgb}{0,0.6,0}
\definecolor{gray}{rgb}{0.5,0.5,0.5}
\definecolor{mauve}{rgb}{0.58,0,0.82}

\lstset{frame=tb,
	language=C++,
	aboveskip=3mm,
	belowskip=3mm,
	showstringspaces=false,
	columns=flexible,
	basicstyle={\small\ttfamily},
	numbers=none,
	numberstyle=\tiny\color{gray},
	keywordstyle=\color{blue},
	commentstyle=\color{dkgreen},
	stringstyle=\color{mauve},
	breaklines=true,
	breakatwhitespace=true,
	tabsize=3
}
\chapter{Metodologia e Desenvolvimento} \label{Cap:Processamento}


Passos do projeto:
\begin{itemize}
	\item Aquisição de imagens em vídeo
			\subitem Nesse passo as imagem são adquiridas via câmera USB.
	\item Identificação de Objetos-Cores
			\subitem Durante o processo de aquisição de imagem são selecionados os objetos-cores, quais serão como base para a detecção de máximos e mínimos de cores.
	\item Calculo de Mínimos e Máximos
		\subitem Nessa etapa são levados em consideração os objetos teste. A imagem é "varida"  por pixel na localidade dos objetos-teste e assim são salvos seus valores e feito a contagem de ocorrências de cada cor.		
%	\item Aplicação de valos para seleção manual de minimo e máximo*

\end{itemize}
\section{Objetivo Especifico}
\subsection{Aquisição da Imagem}
%\begin{lstlisting}
%// Hello.java
%import javax.swing.JApplet;
%import java.awt.Graphics;
%
%public class Hello extends JApplet {
%public void paintComponent(Graphics g) {
%g.drawString("Hello, world!", 65, 95);
%}    
%}
%\end{lstlisting}
\subsection{Seleção de Cores}
\subsection{Definição de Mínimos e Máximos}


%\subsection{Detecção de Objetos em Imagem Estática}
%\subsection{Detecção de Objetos baseado em histograma de cores}	
%\subsection{Detecção de Objetos em Tempo Real}
%\subsection{Detecção de Objetos Dinâmicos em Tempo Real}
\section{Objetivos Específicos} 
\subsection{Implementação de Interface}
\subsection{Calibração Inteligente}
\newpage

	
\end{document}
