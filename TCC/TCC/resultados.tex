%====================================================================================================
% ?????
%====================================================================================================
% TCC
%----------------------------------------------------------------------------------------------------
% Autor				: Jasane Schio
% Orientador		: Gedson Faria
% Co-Orientador		: Angelo Darcy
% Instituição 		: UFMS - Universidade Federal do Mato Grosso do Sul
% Departamento		: CPCX - Sistema de Informação
%----------------------------------------------------------------------------------------------------
% Data de criação	: 01 de Outubro de 2015
%====================================================================================================
% NO FUTURO
\chapter{Resultados} 

Para obtenção dos resultados foram organizados dois tipos de testes: Teste por região e
Teste geral.

No teste por região o campo foi dividido verticalmente em cinco faixas e cada faixa dividia em três partes horizontais. Cada uma das faixas possue ao seu centro um conjunto de cinco cores.
Para cada uma das faixas foram testadas calibrações com cinco tipos diferentes de contraste e luminosidade, somando um total de vinte e cinco testes.
Cada faixa horizontal possui vinte e nove centimetros, e cada faixa vertical possui quarenta e um centimetros, com um intervalo de um centimetro entre elas.
No total estavam disposto no campo quinze pontos de cada cor, sendo as cores: Amarelo, Azul, Laranja, Verde e Vermelho.
Sendo cinco areas horizontais e três verticais, a divisão do campo resulta em quinze partes, cada um das partes sera testada vinte e cinco vezes.


\begin{table}[h]
\centering
\caption{Tabela de Testes}
\begin{tabular}{r|lr}
 
Tipo de Teste & Quantidade \\ % Note a separação de col. e a quebra de linhas
\hline                               % para uma linha horizontal
Divisões no Campo        & 25 \\
Constrate  & 5\\
Brilho            & 5 \\
\hline  
Total & 260
 
\end{tabular}
\end{table}
