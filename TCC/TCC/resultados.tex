%====================================================================================================
% ?????
%====================================================================================================
% TCC
%----------------------------------------------------------------------------------------------------
% Autor				: Jasane Schio
% Orientador		: Gedson Faria
% Co-Orientador		: Angelo Darcy
% Instituição 		: UFMS - Universidade Federal do Mato Grosso do Sul
% Departamento		: CPCX - Sistema de Informação
%----------------------------------------------------------------------------------------------------
% Data de criação	: 01 de Outubro de 2015
%====================================================================================================
% NO FUTURO
\chapter{Resultados} 


	Para serem feitos os testes foram dispostos no campo tiras coloridas entre 17 e 40 centímetros de largura e  5,5 e 10,5 centímetros de altura. Cada cor possue três tiras, e cada uma das tiras foi colocada em uma parte do campo que foi dividio em três.


O campo foi dividido verticalmente em cinco partes, de vinte e nove centímetros cada, e horizontalmente em três parte de quarenta e um centímetro cada, somando um total de quinze áreas de calibração, nomeadas alfabeticamente de A à O, como mostrado na Figura \ref{campodivisao}.

\begin{figure}[!: h]
		\centering
		\includegraphics[width=0.3\textwidth]{campodivisao.pdf}
		\caption{Divisao do campo em quinze partes nomeadas alfabeticamente.}
		\label{campodivisao}
	\end{figure}
	
Em cada áreas de calibração estão dispostas sete cores: Vermelho, Amarelo e Azul na primeira linha, Verde, Roxo, Laranja e Rosa na segunda. As cores estão distantes verticalmente \textit{6cm}, na primeria linha a distancia entre as cores é de \textit{7,25 cm} e na segunda linha de \textit{5 cm}. Um  melhor detalhamento da disposiçao das cores é  mostrado na Figura \ref{disposicaoparte}. No total estavam disposto no campo quinze pontos de cada cor.



%\begin{figure}[H]
%		\centering
%		\includegraphics[width=0.3\textwidth]{disposicaoparte.pdf}
%		\caption{Disposição dos objetos coloridos dentro de cada uma das partes}
%		\label{disposicaoparte}
%	\end{figure}
\begin{figure}[H]
\begin{minipage}[b]{0.45\linewidth}
\centering
\includegraphics[width=\textwidth]{disposicaoparte.pdf}
\caption{Disposiçao de cada parte quanto as cores}
\label{fig:figure1}
\end{minipage}
\hspace{0.5cm}
\begin{minipage}[b]{0.45\linewidth}
\centering
\includegraphics[width=\textwidth]{/testes/campofundo.pdf}
\caption{Campo apos terem sido dispostas as cores}
\label{fig:figure2}
\end{minipage}
\end{figure}

	
A realização da aquisição de mínimos e máximos ocorreu dia no 19 de Agosto de 2016, entre 17:36 e 17:39.

Seleção de Campo:
\begin{figure}[H]
		\centering
		\includegraphics[width=0.6\textwidth]{fundoteste.pdf}
		\caption{Imagem do fundo com a seleção de campo}
		\label{disposicaoparte}
	\end{figure}

	
	\begin{figure}[H]
\begin{minipage}[H]{0.34\linewidth}
\hspace{0.5cm}
\centering
\includegraphics[width=\textwidth]{objetosdispostos.pdf}
\caption{Objetos dispostos no campo para calibração}
\label{fig:figure1}
\end{minipage}
\hspace{0.5cm}
\begin{minipage}[H]{0.40\linewidth}
\centering
\includegraphics[width=\textwidth]{mascaragerada.pdf}
\caption{Mascara gerada a partir da subtração de fundo}
\label{fig:figure2}
\end{minipage}
\end{figure}	
	
	
Arquivo \textit{cores.arff} gerado:
\begin{center}
21.50.50 \newline
30.255.255\newline
92.100.100\newline
120.255.255\newline
62.100.100\newline
90.255.255\newline
169.100.100\newline
179.255.255\newline
0.100.100\newline
20.255.255\newline
161.100.100\newline
168.255.255\newline
126.30.30\newline
160.255.255\newline
\end{center}







 



\section{Amarelo}
	\begin{figure}[H]
		\centering
		\includegraphics[width=0.5\textwidth]{/testes/amarelo.pdf}
		\caption{Imagem somente com os objetos dentro do intervalo do valor da cor amarela}
		\label{disposicaoparte}
	\end{figure}
	
	Dentre os objetos da cor amarela o sistema encontrou quatorze deles completamente e apenas um, que devido a luminosidade implicada em seu centro deixando a tonalizadade muito perto do branco, não totalmente preenchido.
	
	\begin{table}[h]
\centering
\begin{tabular}{l|c|c}
Tipo de Objeto & Quantidade & \%  \\% Note a separação de col. e a quebra de linhas
\hline                               % para uma linha horizontal
Objetos Completos & 14 & 93,33 \\
\hline 
Objetos Com Falha de Preenchimento & 1 & 6,66 \\
\hline 
Objetos Com Diminuição de Contorno & 0 &\\
\hline 
Objetos Extrapolados & 0 &\\
\hline 
Objetos Com Diminuição de Área &  0 &\\
\hline 
Objetos Com Falhas Problematicas & 0 &\\
\hline 
\end{tabular}
\caption{Categorizaçao Dos Objetos}
\end{table}

Dentre as classficações dos objetos as que  \textit{Objetos Com Falhas Problematicas} e \textit{Objetos Com Falha de Preenchimento} que juntos somam 6,67\% dos objetos.

\section{Azul}
	\begin{figure}[H]
		\centering
		\includegraphics[width=0.5\textwidth]{/testes/azul.pdf}
		\caption{Imagem somente com os objetos dentro do intervalo do valor da cor azul}
		\label{disposicaoparte}
	\end{figure}

Os objetos da cor azul foram os que obtiveram os melhores resultados, os quinze objetos foram encontrados de forma preenchida.	Apesar dos quinze estarem totalmente preenchidos um dos objetos apresentou um tamanho reduzido aos demais devido ao fato de sua borda que não ter sido totamente detectada.

\begin{table}[h]
\centering
\begin{tabular}{l|c|c}
Tipo de Objeto & Quantidade & \% \\ % Note a separação de col. e a quebra de linhas
\hline                               % para uma linha horizontal
Objetos Completos & 14 & 93,33 \\
\hline 
Objetos Com Falha de Preenchimento & 0\\
\hline 
Objetos Com Diminuição de Contorno &  1 & 6,66
\\
\hline 
Objetos Extrapolados &  0\\
\hline 
Objetos Com Diminuição de Área & 0 \\
\hline 
Objetos Com Falhas Problematicas & 0 \\
\hline 
\end{tabular}
\caption{Categorizaçao Dos Objetos}
\end{table}

\section{Verde}
	\begin{figure}[H]
		\centering
		\includegraphics[width=0.5\textwidth]{/testes/verde.pdf}
		\caption{Imagem somente com os objetos dentro do intervalo do valor da cor verde}
		\label{disposicaoparte}
	\end{figure}

Os objetos da cor verde foram satisfatoriamente encontrados, com seu preenchimento total e não havendo perda de área devido a qualquer interferência de luz em sua borda.	
	
\begin{table}[h]
\centering
\begin{tabular}{l|c|c}
Tipo de Objeto & Quantidade & \% \\ % Note a separação de col. e a quebra de linhas
\hline                               % para uma linha horizontal
Objetos Completos &  15 &100 \\
\hline 
Objetos Com Falha de Preenchimento & 0\\
\hline 
Objetos Com Diminuição de Contorno &  0\\
\hline 
Objetos Extrapolados & 0 \\
\hline 
Objetos Com Diminuição de Área &  0 \\
\hline 
Objetos Com Falhas Problematicas & 0 \\
\hline 
\end{tabular}
\caption{Categorizaçao Dos Objetos}
\end{table}	
	
\section{Vermelho}
	\begin{figure}[H]
		\centering
		\includegraphics[width=0.5\textwidth]{/testes/vermelho.pdf}
		\caption{Imagem somente com os objetos dentro do intervalo do valor da cor vermelho}
		\label{disposicaoparte}
	\end{figure}
	
	\begin{table}[h]
\centering
\begin{tabular}{l|c|c}
Tipo de Objeto & Quantidade  & \% \\ % Note a separação de col. e a quebra de linhas
\hline                               % para uma linha horizontal
Objetos Completos &  8 & 53,33 \\
\hline 
Objetos Com Falha de Preenchimento & 1 & 6,66 \\
\hline 
Objetos Com Diminuição de Contorno &  4 & 26,66 \\
\hline 
Objetos Extrapolados &  13 \\
\hline 
Objetos Com Diminuição de Área &  1 &6,66 \\
\hline 
Objetos Com Falhas Problematicas &  1 & 6,66\\
\hline 
\end{tabular}
\caption{Categorizaçao Dos Objetos}
\end{table}

\section{Rosa}
	\begin{figure}[H]
		\centering
		\includegraphics[width=0.5\textwidth]{/testes/rosa.pdf}
		\caption{Imagem somente com os objetos dentro do intervalo do valor da cor rosa}
		\label{disposicaoparte}
	\end{figure}
	
Dentre os objetos rosa detectados, quatro obtiveram falhas em sua detecção, falhas de preenchimento e diminuição de borda, estes podem ser desconsiderados dos objetos. Dentro os outros onze: três foram detectados sem falhas de preenchimentos apenas com diminuição de sua área para aproximadamente metade da área real do objeto; os outros oito foram encontrados apesas com diminuição de área devido a diminuição de borda.
	
	\begin{table}[h]
\centering
\begin{tabular}{l|c|r}
Tipo de Objeto & Quantidade  & \% \\ % Note a separação de col. e a quebra de linhas
\hline                               % para uma linha horizontal
Objetos Completos &  0\\
\hline 
Objetos Com Falha de Preenchimento & 0\\
\hline 
Objetos Com Diminuição de Contorno & 8& 53,33
 \\
\hline 
Objetos Extrapolados & 0 \\
\hline 
Objetos Com Diminuição de Área & 3 & 20\\
\hline 
Objetos Com Falhas Problematicas & 4 & 26,66 \\
\hline 
\end{tabular}
\caption{Categorizaçao Dos Objetos}
\end{table}

\section{Roxo}
	\begin{figure}[H]
		\centering
		\includegraphics[width=0.5\textwidth]{/testes/roxo.pdf}
		\caption{Imagem somente com os objetos dentro do intervalo do valor da cor roxo}
		\label{disposicaoparte}
	\end{figure}

Todos os objetos roxos dispostos no campo foram encontrados pelo intervalo da cor. Dentre os quinze, quatro não apresentaram problema algum e foram completamente detectados; oito possuiram diminiução em seu contorno e três diminuição em sua área relativa.
\begin{table}[h]
\centering
\begin{tabular}{l|c|r}
Tipo de Objeto & Quantidade  & \% \\ % Note a separação de col. e a quebra de linhas
\hline                               % para uma linha horizontal
Objetos Completos &  4 & 26,66\\
\hline 
Objetos Com Falha de Preenchimento & 0 \\
\hline 
Objetos Com Diminuição de Contorno &  8 & 53,33\\
\hline 
Objetos Extrapolados & 0 \\
\hline 
Objetos Com Diminuição de Área & 3 & 20\\
\hline 
Objetos Com Falhas Problematicas & 0 \\
\hline 
\end{tabular}
\caption{Categorizaçao Dos Objetos}
\end{table}
	
\section{Laranja}
	\begin{figure}[H]
		\centering
		\includegraphics[width=0.5\textwidth]{/testes/laranja.pdf}
		\caption{Imagem somente com os objetos dentro do intervalo do valor da cor laranja}
		\label{disposicaoparte}
	\end{figure}
	
	Os objetos laranjas foram intencionalmente deixados por último.
	Devido à um problema muito comum na área de calibração de cores, a cor laranja possui o problema de ser, por muitas vezes, semelhante a vermelha, e devido a luminosidade implicada tanto em uma quanto na outra cor, ambas tendem a se tornarem próximas.
	Sabendo deste problema, o fato de terem sido encontrados objetos da cor vermelha dentro do intervalo de valores da cor laranja é ignorado e somente serão levados em consideração os objetos realmente laranjas.
	Os quinze objetos da cor laranja foram encontrados com precisão. Todos possuindo seu completo preenchimento e borda.
	
\begin{table}[h]
\centering
\begin{tabular}{l|c|r}
Tipo de Objeto & Quantidade  & \% \\ % Note a separação de col. e a quebra de linhas
\hline                               % para uma linha horizontal
Objetos Completos &  15 & 100 \\
\hline 
Objetos Com Falha de Preenchimento & 0 \\
\hline 
Objetos Com Diminuição de Contorno &  0 \\
\hline 
Objetos Extrapolados & 8 \\
\hline 
Objetos Com Diminuição de Área &  0 \\
\hline 
Objetos Com Falhas Problematicas & 0 \\
\hline 
\end{tabular}
\caption{Categorizaçao Dos Objetos}
\end{table}
\newpage
\section{Totais}
De modo total estavam disposto pelo campo cento e cinco objetos coloridos. Destes, 66,7\% dos objetos foram encontrados corretamente, 1,9\% foram encontrados com falhas de preenchimento, 20\% apresentaram perda de contorno, 6,67\% apresentaram diminuição da área e 4,76\% apresentaram falhas e faltas que prejudicaram totalmente o objeto.
	\begin{figure}[H]
		\centering
		\includegraphics[width=0.8\textwidth]{/testes/graficotestes.pdf}
		\caption{Grafico de analise do resultado dos testes}
		\label{disposicaoparte}
	\end{figure}
	
	
	Sabendo destes valores, os que podem ser considerados criticos na detecção de objetos atraves da sua cor são \textit{Objetos Com Falhas Problematicas} e \textit{Objetos Com Falha de Preenchimento} que juntos somam 6,67\% dos objetos.