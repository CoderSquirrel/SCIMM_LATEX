%====================================================================================================
% ?????
%====================================================================================================
% TCC
%----------------------------------------------------------------------------------------------------
% Autor				: Jasane Schio
% Orientador		    : Gedson Faria
% Co-Orientador		: Angelo Darcy
% Instituição 		: UFMS - Universidade Federal do Mato Grosso do Sul
% Departamento		: CPCX - Sistema de Informação
%----------------------------------------------------------------------------------------------------
% Data de criação	: 01 de Outubro de 2015
%====================================================================================================

\chapter*{Resumo}

	 Para serem identificadas as tiras a serem calibrados no campo, utilizou-se a técnica Algoritmo de Canny, que faz a detecção das bordas presentes na imagem. Após serem detectadas as bordas fez-se então uma analise para identificar quais bordas pertencem à mesma tira, é somente após essa analise que as tiras são identificas pela junção de suas respectivas bordas. Assim que identificadas, cada um dos tiras foi analisado de acordo com o estudo das cores para assimilar a qual cor a mesma pertencia, e seus valores comparados para identificar se esta seria um limite do intervalo da cor a qual pertence.
