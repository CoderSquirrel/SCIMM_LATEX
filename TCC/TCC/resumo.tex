%====================================================================================================
% ?????
%====================================================================================================
% TCC
%----------------------------------------------------------------------------------------------------
% Autor				: Jasane Schio
% Orientador		    : Gedson Faria
% Co-Orientador		: Angelo Darcy
% Instituição 		: UFMS - Universidade Federal do Mato Grosso do Sul
% Departamento		: CPCX - Sistema de Informação
%----------------------------------------------------------------------------------------------------
% Data de criação	: 01 de Outubro de 2015
%====================================================================================================

\chapter*{Resumo}

Devido a aspectos como fontes de luz, sombras e luminosidade, uma única cor pode apresentar variações dentro de uma mesma imagem. No futebol de robôs, categoria {\it Very Small Size Soccer}, este problema também existe, pois os robôs são detectados por câmera e reconhecidos por marcadores coloridos sobre eles. Sendo assim, um desafio do futebol de robôs é a implementação do processamento digital de imagens para o reconhecimento de objetos coloridos. Neste trabalho é proposto um sistema automático de calibração de cores, reduzindo o tempo gasto pelas equipes de futebol de robôs nesta tarefa. Foram utilizadas a técnica Algoritmo de Canny, que faz a detecção das bordas presentes na imagem e em seguida utiliza-se do algoritmo {\it border following}, ambos presentes na biblioteca OpenCV. Como resultado as cores verde e laranja obtiveram 100\% de sucesso na detecção, a cores amarelo e azul foram detectadas com sucesso com percentual de erro de 6,66\%, a cor roxa teve todos os objetos encontrados com 26,66\% deles encontrados completamente, a cor vermelha obteve sucesso da detecção de seus objetos com percentual de erro 19,99\%, e a cor Rosa não foi detectada com sucesso.


Palavras-chave: Calibração de Cores, Futebol de Robôs, OpenCV, Visão Computacional, HSV, Equipe Cedro;
